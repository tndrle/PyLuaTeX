%% Copyright 2021-2022 Tobias Enderle
%%
%% This work may be distributed and/or modified under the
%% conditions of the LaTeX Project Public License, either version 1.3
%% of this license or (at your option) any later version.
%% The latest version of this license is in
%%   http://www.latex-project.org/lppl.txt
%% and version 1.3 or later is part of all distributions of LaTeX
%% version 2005/12/01 or later.

\documentclass{beamer}

\usepackage{pyluatex}
\usepackage{listings}
\usepackage{xcolor}
\lstset{
    language=Python,
    breaklines=true,
    framesep=1ex,
    frame=lrtb,
    framerule=0pt,
    numbers=none,
    basicstyle=\ttfamily,
    keywordstyle=\bfseries\color{green!40!black},
    stringstyle=\bfseries\color{red!80!black},
    identifierstyle=\color{blue},
    backgroundcolor=\color{gray!10!white},
}

\usepackage{luacode}
\begin{luacode}
function pytypeset()
    tex.print("\\begin{lstlisting}[language=Python]")
    tex.print(pyluatex.get_last_code())
    tex.print("\\end{lstlisting}")
    tex.print("") -- ensure newline
end

function pytypeset_inline()
    -- assume there is only one line of code in get_last_code()
    tex.print("\\lstinline[columns=fixed]@" .. pyluatex.get_last_code()[1] .. "@")
end
\end{luacode}

\newcommand*{\pytypeset}{%
    \textbf{Input:}
    \directlua{pytypeset()}
    \textbf{Output:}
    \begin{center}
        \directlua{tex.print(pyluatex.get_last_output())}
    \end{center}
}
\newcommand*{\coderaw}{\directlua{tex.print(pyluatex.get_last_code())}}
\newcommand*{\codeinline}{\directlua{pytypeset_inline()}}
\newcommand*{\outputraw}{\directlua{tex.print(pyluatex.get_last_output())}}

\title{PyLuaTeX Example -- BEAMER Presentation}
\author{Tobias Enderle}

\begin{document}

\maketitle

%%%%%%%%%%%%%%%%%%%%%% frame

\begin{frame}{Important}
Using PyLuaTeX environments (\emph{python} or \emph{pythonq}) inside BEAMER frames
requires the \emph{fragile} option for those frames.
\end{frame}

%%%%%%%%%%%%%%%%%%%%%% frame

\begin{frame}[fragile]{Slide 1}
\begin{pythonq}
msg = 'Hello slide 1'

print(msg)
\end{pythonq}
\pytypeset

Calling \pycq{print('test', end='')}\codeinline\ in Python outputs ``\outputraw''.

\begin{python}
x = 4
\end{python}
The value of \pyq{x}\codeinline\ is \outputraw.

The result of \pyq{17 + 300}$\coderaw$ is \outputraw.
\end{frame}

%%%%%%%%%%%%%%%%%%%%%% frame

\begin{pythonq}
msg = 'Hello slide 2'

print(msg)
\end{pythonq}

\begin{frame}{Slide 1}
\pytypeset

The \emph{fragile} option is not required in this frame because
the \emph{pythonq} environment is outside the frame.
Only the macro \emph{\textbackslash pytypeset} for typesetting the code and output is inside
the frame.
\end{frame}

%%%%%%%%%%%%%%%%%%%%%% frame

\newenvironment{typesetpython}
    {\PyLTVerbatimEnv\begin{pythonq}}
    {\end{pythonq}\pytypeset}

\begin{frame}[fragile]{Slide 2}
\begin{typesetpython}
msg = 'Custom environment'

print(msg)
\end{typesetpython}
\end{frame}

\end{document}
